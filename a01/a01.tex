\documentclass[12pt, a4paper, oneside, ngerman]{article}

\newif\ifhpcGerman
\hpcGermanfalse

\usepackage[utf8]{inputenc}
\usepackage{geometry}
\geometry{a4paper, left=25mm, right=20mm, top=2.35cm, bottom=2.35cm}

\usepackage[T1]{fontenc}
\usepackage{totcount}
\usepackage{calc}
\usepackage{graphicx}
\usepackage{ifthen}
\usepackage[usenames]{color}
\usepackage{caption}
\usepackage{multicol}
\usepackage{lipsum}
\usepackage{listings}
\usepackage{courier}
%\usepackage[nolist]{acronym}
\usepackage[acronym]{glossaries} % do not use hyperref! that's why we include it before
\usepackage[hyphens]{url}
\usepackage[pdfborder={0 0 0}, colorlinks=true,urlcolor=blue,linkcolor=blue,citecolor=blue]{hyperref}
\usepackage{microtype}
\usepackage{tabu}
\usepackage{xspace}
\usepackage{fix-cm}
%\usepackage{pdflscape}
\usepackage{lscape}
\usepackage{enumitem}
%\usepackage{enumerate}
\usepackage{amsmath}
\usepackage{amssymb}
\usepackage{wasysym}
\usepackage[detect-weight=true, detect-family=true]{siunitx}
\usepackage{xfrac}
\usepackage{mathtools}
\newcommand{\abs}[1]{ \left\lvert#1\right\rvert} % absolute value: single vertical bars
\newcommand{\norm}[1]{\left\lVert#1\right\rVert} % norm: double vertical bars

\ifhpcGerman % GERMAN STUFF
\usepackage[ngerman]{babel}
\usepackage[babel, german=quotes]{csquotes}
%\usepackage[babel, german=guillemets]{csquotes}
\else
%\usepackage[english=british]{csquotes}
\usepackage[english=american]{csquotes}
\fi

\newif\ifhpcBonus % enables bonus points
\hpcBonusfalse
\newif\ifhpcFake % toggles preparatory (fake) and final exam (real)
\hpcFakefalse
\newif\ifhpcKnown % toggles known student name, matriculation no. etc.
\hpcKnownfalse
\newif\ifhpcSolution % toggles solution
\hpcSolutionfalse

\newcommand{\hpcGerman}[2]{\ifhpcGerman#1\else#2\fi}
\newcommand{\hpcBonus}[2]{\ifhpcBonus#1\else#2\fi}
\newcommand{\hpcFake}[2]{\ifhpcFake#1\else#2\fi}
\newcommand{\hpcKnown}[2]{\ifhpcKnown#1\else#2\fi}

% The solution commands take always two arguments. The solution text that should be displayed
% when solution mode is enabled and the other text that should be displayed in the normal mode.
% Use simple "\s{<solution text>}{<normal text>} for bold colored solutions.
\newcommand{\s}[2]{\ifhpcSolution\textcolor[rgb]{0.8,0.0,0.8}{\textbf{#1}}\else#2\fi}
%\newcommand{\s}[2]{\ifprintanswers\textcolor[rgb]{0.8,0.0,0.8}{\textbf{#1}}\else#2\fi} % exam solution stuff
% Use "\solution{<solution text>}{<normal text>} for subtle solutions.
\newcommand{\hpcSolution}[2]{\ifhpcSolution#1\else#2\fi}
%\newcommand{\hpcSolution}[2]{\ifprintanswers#1\else#2\fi} % exam solution stuff

\definecolor{dkgreen}{rgb}{0,0.4,0}
\definecolor{dkred}{rgb}{0.5,0,0}
\definecolor{lightgray}{rgb}{0.9,0.9,0.9}
\definecolor{mygray}{rgb}{0.5,0.5,0.5}
\definecolor{mymauve}{rgb}{0.58,0,0.82}

\lstdefinelanguage{out}{
%sensitive=false,
%alsoletter={.},
%moredelim=[s][\color{red}]{<}{>},
%moredelim=[s][\color{blue}]{[}{]},
%moredelim=[is][\color{orange}]{:}{:},
%keywords=[10]{...},
%keywordstyle=[10]{\color{magenta}},
}

\lstnewenvironment{stdout}
{\lstset{numbers=none, language=out, frame=none, backgroundcolor={\color{lightgray}}, basicstyle={\footnotesize\ttfamily}}}
{}

\lstset{
  % the following allows us to handle UTF-8 encoding issues, listings does not support UTF-8 code
  literate=
  {á}{{\'a}}1 {é}{{\'e}}1 {í}{{\'i}}1 {ó}{{\'o}}1 {ú}{{\'u}}1
  {Á}{{\'A}}1 {É}{{\'E}}1 {Í}{{\'I}}1 {Ó}{{\'O}}1 {Ú}{{\'U}}1
  {à}{{\`a}}1 {è}{{\'e}}1 {ì}{{\`i}}1 {ò}{{\`o}}1 {ò}{{\`u}}1
  {À}{{\`A}}1 {È}{{\'E}}1 {Ì}{{\`I}}1 {Ò}{{\`O}}1 {Ò}{{\`U}}1
  {ä}{{\"a}}1 {ë}{{\"e}}1 {ï}{{\"i}}1 {ö}{{\"o}}1 {ü}{{\"u}}1
  {Ä}{{\"A}}1 {Ë}{{\"E}}1 {Ï}{{\"I}}1 {Ö}{{\"O}}1 {Ü}{{\"U}}1
  {â}{{\^a}}1 {ê}{{\^e}}1 {î}{{\^i}}1 {ô}{{\^o}}1 {û}{{\^u}}1
  {Â}{{\^A}}1 {Ê}{{\^E}}1 {Î}{{\^I}}1 {Ô}{{\^O}}1 {Û}{{\^U}}1
  {œ}{{\oe}}1 {Œ}{{\OE}}1 {æ}{{\ae}}1 {Æ}{{\AE}}1 {ß}{{\ss}}1
  {ç}{{\c c}}1 {Ç}{{\c C}}1 {ø}{{\o}}1 {å}{{\r a}}1 {Å}{{\r A}}1
  {€}{{\EUR}}1 {£}{{\pounds}}1,
  %extendedchars=true,
  %backgroundcolor=\color{white},    % choose the background color; you must add \usepackage{color} or \usepackage{xcolor}
  numbers=left,                      % where to put the line-numbers; possible values are (none, left, right)
  numbersep=5pt,                     % how far the line-numbers are from the code
  numberstyle=\scriptsize\ttfamily\color{mygray},   % the style that is used for the line-numbers
  language=Java,                     % the language of the code
  basicstyle=\scriptsize\ttfamily, % the size of the fonts that are used for the code
  showspaces=false,                  % show spaces adding particular underscores
  showstringspaces=false,            % underline spaces within strings
  showtabs=false,                    % show tabs within strings adding particular underscores
  showlines=true,                    % show empty lines at the end of the listing
  %stepnumber=2,                     % the step between two line-numbers. If it's 1, each line will be numbered
  frame=leftline,                    % adds a frame around the code
  rulecolor=\color{black},           % if not set, the frame-color may be changed on line-breaks within not-black text (e.g. commens (green here))
  tabsize=2,                         % sets default tabsize to 2 spaces
  stringstyle=\color{mymauve},       % string literal style
  captionpos=b,                      % sets the caption-position to bottom
  breaklines=true,                   % sets automatic line breaking
  breakatwhitespace=false,           % sets if automatic breaks should only happen at whitespace
  keywordstyle=\color{blue},         % keyword style
  commentstyle=\color{dkgreen},      % comment style
  stringstyle=\color{dkred},         % string literal style
  escapeinside={\%*}{*)},            % if you want to add LaTeX within your code
  %title=\lstname,                   % show the filename of files included with \lstinputlisting; also try caption instead of title
  morekeywords={in, vec3, vec4, *,...} % if you want to add more keywords to the set
}
\lstloadlanguages{Java}
\DeclareCaptionFont{green}{\color{green}}
\DeclareCaptionFont{white}{\color{white}}
\DeclareCaptionFormat{listing}{\colorbox[cmyk]{0.43, 0.35, 0.35,0.01}{\parbox{\textwidth}{\hspace{15pt}#1#2#3}}}

\captionsetup[lstlisting]{format=listing, labelfont=white,
                         textfont=white, singlelinecheck=false, margin=0pt,
                         font={footnotesize}}



\newcommand{\hpcSheetNumber}{1}
\newcommand{\hpcDeadline}{April 28, 2021 at 11:55pm}

%%% define class and exam details in header file!

%\hpcSolutiontrue

\begin{document}

\rmfamily

\newtotcounter{totalPoints}

\newcommand{\Moodle}{\href{https://moodle2.uni-due.de/course/view.php?id=10062}{Moodle}\xspace}
\newcommand{\hpcClass}{Advanced Image Synthesis\xspace}
\newcommand{\hpcTerm}{Summer Term 2021\xspace}
\newcommand{\hpcPeople}{\textbf{Andrea Iannella}\xspace}

%\newcommand{\class}[1]{\texttt{#1}}

% cover page header
\newcommand{\hpcHeader}{
\noindent
\begin{minipage}[b][3.5cm]{0.39\textwidth}
	\includegraphics[width=6cm]{../uni-due}\\
	\vfill
%	\hpcGerman{
%	\s{Lösung}{} \hpcFake{Probeklausur}{Klausur} \hpcTerm\
%	}{
%	\s{Solution}{} \hpcFake{Preparatory Exam}{Final Exam} \hpcTerm\
%	}
	\hpcTerm
	%\\ \hpcExamDate
\end{minipage}
\hfill
\begin{minipage}[b][3.5cm]{0.59\textwidth}
\begin{flushright}
	\hpcGerman{
	Informatik und Angewandte Kognitionswissenschaft\\
	Lehrstuhl für Hochleistungsrechnen\\
	}{
	Computer Science and Applied Cognitive Science\\
	High Performance Computing Group\\
	}
	\url{http://hpc.uni-due.de/teaching}\\
	\vfill
	\textbf{Prof. Dr. Jens Krüger}\\
	\hpcPeople
\end{flushright}
\end{minipage}
\\
\vspace{-3.5mm}
\hrule
}

\hpcHeader

\let\origthesubsection\thesubsection
\setcounter{section}{0}

\vspace{0.5cm}

\begin{center}
	\LARGE
	\textbf{\hpcClass}\\
\end{center}

\vspace{0.2cm}

\ifhpcGerman % GERMAN STUFF

\normalfont
\begin{center}
	\Large
	{\textbf{\hpcSolution{Musterlösung für}{} Übungsblatt \hpcSheetNumber \ (\total{totalPoints} Punkte)}}
\end{center}

\vspace{0.2cm}

\hpcSolution{
\hrule height 0.07ex
}{
Alle Abgaben müssen in \Moodle bis \textbf{\hpcDeadline{}} hochgeladen werden, anderenfalls wird das Übungsblatt mit 0 Punkten bewertet. Abgaben, die nicht kompilierbar sind oder falsch benannte Dateien enthalten, bekommen ebenfalls 0 Punkte.
Beispielhafte (Konsolen-)Ausgaben sind häufig Teil der Aufgabe. Stimmen Ihre Lösungen nicht mit den beispielhaften Ausgaben überein, erhalten Sie nur sehr wenige Punkte. Bitte fügen Sie keine weiteren, nicht in der Aufgabe geforderten Ausgaben hinzu.
Packen Sie alle Dateien Ihrer Lösung in \textbf{eine}
\href{http://de.wikipedia.org/wiki/ZIP-Dateiformat}{Zip-Datei}.
Innerhalb dieser Zip-Datei dürfen keine Unterverzeichnisse angelegt werden. Der Dateiname muss wie folgt aussehen: \textbf{\texttt{u\hpcSheetNumber\_Vorname\_Nachname.zip}}.
}

\newcommand{\ex}[2]{
\addtocounter{totalPoints}{#2}
\section{#1 (#2 Punkte)}
}

\else % ENGLISH STUFF

\normalfont
\begin{center}
	\Large
	{\textbf{\hpcSolution{Sample Solution for}{} Assignment \hpcSheetNumber \ (\total{totalPoints} Points)}}
\end{center}

\vspace{0.2cm}

\hpcSolution{
\hrule height 0.07ex
}{
All assignments are to be uploaded to \Moodle on \textbf{\hpcDeadline{}}.
Assignments that are not uploaded to Moodle before the deadline will
not be graded (0 points). Assignments that do not compile also
receive 0 points.
Often example inputs and outputs are provided for you
to test your programs; assignments that do not process these examples
correctly will receive very few points.
Please do not add any additional output to your program than what is
requested.
Submit your solution as a \textbf{single} \href{http://en.wikipedia.org/wiki/Zip_(file_format)}{zip-archive} named as follows: \textbf{\texttt{a\hpcSheetNumber\_FirstName\_LastName.zip}}.

Include the entire project folder in the archive and for assignments also consisting of theoretical tasks include in addition a text file or a scan/photo with the corresponding answers.
Please see assignment sheet 1 for details.
}

\newcommand{\ex}[2]{
\addtocounter{totalPoints}{#2}
\section{#1 (#2 Points)}
}

\fi

\vspace{1em}
%\hline
\vspace{1em}


\newacronym{vbo}{VBO}{vertex buffer object}
\newacronym{vao}{VAO}{vertex array object}
\newacronym{glsl}{GLSL}{OpenGL shading language}

\section{How to get started}

Assignments consist of theoretical and programming tasks. The programming tasks require a C++ environment.\\

In order to get you started for the assignments, please follow these steps to setup your programming environment. \\
\\
\textbf{For Windows users:}

\begin{enumerate}
\item You should download the latest \href{https://visualstudio.microsoft.com}{Visual Studio IDE} (NOT Visual Studio Code). This can be downloaded for free from Microsoft. The Community Edition is sufficient for our needs.

\item Run the installer and check 'Desktop development with C++' in the 'Desktop \& Mobile' section of the 'Workloads' tab. Proceed with the installation.

\item Download the provided source code archive corresponding to the assignment sheet from Moodle, decompress it and use the .sln file to open the solution with Visual Studio.\\

\end{enumerate}
\textbf{For Mac users:}\\

You should install Xcode. It can be downloaded from the app store for free. Also install \href{http://glew.sourceforge.net/}{glew} and \href{https://www.glfw.org/}{glfw}.\\
\\
\textbf{For Linux users:}\\

You should already have all necessary tools installed on your system. In the rare situation that you installed only a minimal distribution and it lacks the build essentials you can install the necessary tools by running \texttt{sudo apt install build-essential}


\section{How to submit your solution}
\label{sec:rules}
For a successful submission all you need to do is to create a \href{http://en.wikipedia.org/wiki/Zip_(file_format)}{zip-archive} of the files containing your answers to the theoretical tasks and the project folder of the programming solution. Submit this archive following the naming convention shown in the header of every assignment sheet. Make sure that your code compiles otherwise you won't receive credit for the programming task.

\textbf{Do not change any helper classes in the provided \texttt{Utils} folder!}

\ex{Hello OpenGL}{20}

This and the following assignments make use of OpenGL for rendering a given scene. All OpenGL exercises need to be written against the \textbf{OpenGL 3.2 core profile or greater}. OpenGL 3 is the basis for \textit{modern} OpenGL programming where all fixed function pipeline and immediate mode OpenGL commands are deprecated.

Consult the reading list at the end of this document for further information about OpenGL. The OpenGL \enquote{\textit{Red Book}} \cite{RedBook} is the recommended way to learn \textit{modern} OpenGL and \cite{RTR3} is a very good compendium about rasterization-based Computer Graphics including chapters introducing the programmable graphics pipeline and math prerequisites. \cite{OglMan} and \cite{OglDoc} are the sources for details about specific OpenGL and \gls{glsl} calls. The \href{http://www.khronos.org/files/opengl-quick-reference-card.pdf}{quick reference cards} found at \cite{OglDoc} for all OpenGL and \gls{glsl} commands starting with OpenGL version 3.2 also give a good overview for experienced OpenGL programmers. 

The \href{http://www.realtech-vr.com/glview}{OpenGL Extensions Viewer} is a handy tool to quickly check the OpenGL capabilities of your machine to verify that your hardware and installed driver support at least version 3.2 or greater.\\

The provided example program \texttt{Hello OpenGL} renders a simple white triangle as shown on the left in Figure~\ref{fig:triangles}. Extend this program so that its output matches \textbf{exactly} the rendering on the right in the same figure that shows two triangles forming a rectangle on a dark cyan background. \\

To complete this task first familiarize yourself with the concept of shaders, \glspl{vbo}, and \glspl{vao}. Change the provided data stored in \texttt{triangle} such that it forms one of the two triangles. Add additional data for the second triangle. To color one of the triangles in a different color than the other one create an additional shader program that uses a different fragment shader code but still the provided vertex shader code.

The normalized color values of the two triangles and background are as follows:
\begin{align*}
    orange(0.9, 0.5, 0.2) \quad
    green(0.2, 0.9, 0.2) \quad
    background(0.1, 0.15, 0.15)
\end{align*}

Get used to calling \texttt{glGetError} after every OpenGL command to quickly identify bad commands with wrong arguments because an OpenGL application's behavior is undefined when it is not conformant to the OpenGL specification. This can lead to inconsistent program behavior that could be observed quite differently across different vendors and platforms. 



To achieve this first change the data

\begin{figure}
  \centering
  \includegraphics[width=0.4875\linewidth]{SimpleTriangle.png}
  \includegraphics[width=0.4875\linewidth]{rectangle.png}
  \caption{Rendering of a simple white triangle (left) and the solution to this task (right).}
  \label{fig:triangles}
\end{figure}

% Publications from BibTeX file
\renewcommand{\refname}{Reading list and references}
\nocite{*}
\bibliographystyle{plain}
%\bibliographystyle{natdin}
%\bibliographystyle{dinat}
%\bibliographystyle{gerplain}
\bibliography{literature}

\end{document}
