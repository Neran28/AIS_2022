\documentclass[12pt, a4paper, oneside, ngerman]{article}

\newif\ifhpcGerman
\hpcGermanfalse

\usepackage[utf8]{inputenc}
\usepackage{geometry}
\geometry{a4paper, left=25mm, right=20mm, top=2.35cm, bottom=2.35cm}

\usepackage[T1]{fontenc}
\usepackage{totcount}
\usepackage{calc}
\usepackage{graphicx}
\usepackage{ifthen}
\usepackage[usenames]{color}
\usepackage{caption}
\usepackage{multicol}
\usepackage{lipsum}
\usepackage{listings}
\usepackage{courier}
%\usepackage[nolist]{acronym}
\usepackage[acronym]{glossaries} % do not use hyperref! that's why we include it before
\usepackage[hyphens]{url}
\usepackage[pdfborder={0 0 0}, colorlinks=true,urlcolor=blue,linkcolor=blue,citecolor=blue]{hyperref}
\usepackage{microtype}
\usepackage{tabu}
\usepackage{xspace}
\usepackage{fix-cm}
%\usepackage{pdflscape}
\usepackage{lscape}
\usepackage{enumitem}
%\usepackage{enumerate}
\usepackage{amsmath}
\usepackage{amssymb}
\usepackage{wasysym}
\usepackage[detect-weight=true, detect-family=true]{siunitx}
\usepackage{xfrac}
\usepackage{mathtools}
\newcommand{\abs}[1]{ \left\lvert#1\right\rvert} % absolute value: single vertical bars
\newcommand{\norm}[1]{\left\lVert#1\right\rVert} % norm: double vertical bars

\ifhpcGerman % GERMAN STUFF
\usepackage[ngerman]{babel}
\usepackage[babel, german=quotes]{csquotes}
%\usepackage[babel, german=guillemets]{csquotes}
\else
%\usepackage[english=british]{csquotes}
\usepackage[english=american]{csquotes}
\fi

\newif\ifhpcBonus % enables bonus points
\hpcBonusfalse
\newif\ifhpcFake % toggles preparatory (fake) and final exam (real)
\hpcFakefalse
\newif\ifhpcKnown % toggles known student name, matriculation no. etc.
\hpcKnownfalse
\newif\ifhpcSolution % toggles solution
\hpcSolutionfalse

\newcommand{\hpcGerman}[2]{\ifhpcGerman#1\else#2\fi}
\newcommand{\hpcBonus}[2]{\ifhpcBonus#1\else#2\fi}
\newcommand{\hpcFake}[2]{\ifhpcFake#1\else#2\fi}
\newcommand{\hpcKnown}[2]{\ifhpcKnown#1\else#2\fi}

% The solution commands take always two arguments. The solution text that should be displayed
% when solution mode is enabled and the other text that should be displayed in the normal mode.
% Use simple "\s{<solution text>}{<normal text>} for bold colored solutions.
\newcommand{\s}[2]{\ifhpcSolution\textcolor[rgb]{0.8,0.0,0.8}{\textbf{#1}}\else#2\fi}
%\newcommand{\s}[2]{\ifprintanswers\textcolor[rgb]{0.8,0.0,0.8}{\textbf{#1}}\else#2\fi} % exam solution stuff
% Use "\solution{<solution text>}{<normal text>} for subtle solutions.
\newcommand{\hpcSolution}[2]{\ifhpcSolution#1\else#2\fi}
%\newcommand{\hpcSolution}[2]{\ifprintanswers#1\else#2\fi} % exam solution stuff

\definecolor{dkgreen}{rgb}{0,0.4,0}
\definecolor{dkred}{rgb}{0.5,0,0}
\definecolor{lightgray}{rgb}{0.9,0.9,0.9}
\definecolor{mygray}{rgb}{0.5,0.5,0.5}
\definecolor{mymauve}{rgb}{0.58,0,0.82}

\lstdefinelanguage{out}{
%sensitive=false,
%alsoletter={.},
%moredelim=[s][\color{red}]{<}{>},
%moredelim=[s][\color{blue}]{[}{]},
%moredelim=[is][\color{orange}]{:}{:},
%keywords=[10]{...},
%keywordstyle=[10]{\color{magenta}},
}

\lstnewenvironment{stdout}
{\lstset{numbers=none, language=out, frame=none, backgroundcolor={\color{lightgray}}, basicstyle={\footnotesize\ttfamily}}}
{}

\lstset{
  % the following allows us to handle UTF-8 encoding issues, listings does not support UTF-8 code
  literate=
  {á}{{\'a}}1 {é}{{\'e}}1 {í}{{\'i}}1 {ó}{{\'o}}1 {ú}{{\'u}}1
  {Á}{{\'A}}1 {É}{{\'E}}1 {Í}{{\'I}}1 {Ó}{{\'O}}1 {Ú}{{\'U}}1
  {à}{{\`a}}1 {è}{{\'e}}1 {ì}{{\`i}}1 {ò}{{\`o}}1 {ò}{{\`u}}1
  {À}{{\`A}}1 {È}{{\'E}}1 {Ì}{{\`I}}1 {Ò}{{\`O}}1 {Ò}{{\`U}}1
  {ä}{{\"a}}1 {ë}{{\"e}}1 {ï}{{\"i}}1 {ö}{{\"o}}1 {ü}{{\"u}}1
  {Ä}{{\"A}}1 {Ë}{{\"E}}1 {Ï}{{\"I}}1 {Ö}{{\"O}}1 {Ü}{{\"U}}1
  {â}{{\^a}}1 {ê}{{\^e}}1 {î}{{\^i}}1 {ô}{{\^o}}1 {û}{{\^u}}1
  {Â}{{\^A}}1 {Ê}{{\^E}}1 {Î}{{\^I}}1 {Ô}{{\^O}}1 {Û}{{\^U}}1
  {œ}{{\oe}}1 {Œ}{{\OE}}1 {æ}{{\ae}}1 {Æ}{{\AE}}1 {ß}{{\ss}}1
  {ç}{{\c c}}1 {Ç}{{\c C}}1 {ø}{{\o}}1 {å}{{\r a}}1 {Å}{{\r A}}1
  {€}{{\EUR}}1 {£}{{\pounds}}1,
  %extendedchars=true,
  %backgroundcolor=\color{white},    % choose the background color; you must add \usepackage{color} or \usepackage{xcolor}
  numbers=left,                      % where to put the line-numbers; possible values are (none, left, right)
  numbersep=5pt,                     % how far the line-numbers are from the code
  numberstyle=\scriptsize\ttfamily\color{mygray},   % the style that is used for the line-numbers
  language=Java,                     % the language of the code
  basicstyle=\scriptsize\ttfamily, % the size of the fonts that are used for the code
  showspaces=false,                  % show spaces adding particular underscores
  showstringspaces=false,            % underline spaces within strings
  showtabs=false,                    % show tabs within strings adding particular underscores
  showlines=true,                    % show empty lines at the end of the listing
  %stepnumber=2,                     % the step between two line-numbers. If it's 1, each line will be numbered
  frame=leftline,                    % adds a frame around the code
  rulecolor=\color{black},           % if not set, the frame-color may be changed on line-breaks within not-black text (e.g. commens (green here))
  tabsize=2,                         % sets default tabsize to 2 spaces
  stringstyle=\color{mymauve},       % string literal style
  captionpos=b,                      % sets the caption-position to bottom
  breaklines=true,                   % sets automatic line breaking
  breakatwhitespace=false,           % sets if automatic breaks should only happen at whitespace
  keywordstyle=\color{blue},         % keyword style
  commentstyle=\color{dkgreen},      % comment style
  stringstyle=\color{dkred},         % string literal style
  escapeinside={\%*}{*)},            % if you want to add LaTeX within your code
  %title=\lstname,                   % show the filename of files included with \lstinputlisting; also try caption instead of title
  morekeywords={in, vec3, vec4, *,...} % if you want to add more keywords to the set
}
\lstloadlanguages{Java}
\DeclareCaptionFont{green}{\color{green}}
\DeclareCaptionFont{white}{\color{white}}
\DeclareCaptionFormat{listing}{\colorbox[cmyk]{0.43, 0.35, 0.35,0.01}{\parbox{\textwidth}{\hspace{15pt}#1#2#3}}}

\captionsetup[lstlisting]{format=listing, labelfont=white,
                         textfont=white, singlelinecheck=false, margin=0pt,
                         font={footnotesize}}



\newcommand{\hpcSheetNumber}{2}
\newcommand{\hpcDeadline}{May 12, 2021 at 11:55pm}

%%% define class and exam details in header file!

%\hpcSolutiontrue

\newacronym{vbo}{VBO}{vertex buffer object}
\newacronym{vao}{VAO}{vertex array object}
\newacronym{glsl}{GLSL}{OpenGL shading language}

\begin{document}

\rmfamily

\newtotcounter{totalPoints}

\newcommand{\Moodle}{\href{https://moodle2.uni-due.de/course/view.php?id=10062}{Moodle}\xspace}
\newcommand{\hpcClass}{Advanced Image Synthesis\xspace}
\newcommand{\hpcTerm}{Summer Term 2021\xspace}
\newcommand{\hpcPeople}{\textbf{Andrea Iannella}\xspace}

%\newcommand{\class}[1]{\texttt{#1}}

% cover page header
\newcommand{\hpcHeader}{
\noindent
\begin{minipage}[b][3.5cm]{0.39\textwidth}
	\includegraphics[width=6cm]{../uni-due}\\
	\vfill
%	\hpcGerman{
%	\s{Lösung}{} \hpcFake{Probeklausur}{Klausur} \hpcTerm\
%	}{
%	\s{Solution}{} \hpcFake{Preparatory Exam}{Final Exam} \hpcTerm\
%	}
	\hpcTerm
	%\\ \hpcExamDate
\end{minipage}
\hfill
\begin{minipage}[b][3.5cm]{0.59\textwidth}
\begin{flushright}
	\hpcGerman{
	Informatik und Angewandte Kognitionswissenschaft\\
	Lehrstuhl für Hochleistungsrechnen\\
	}{
	Computer Science and Applied Cognitive Science\\
	High Performance Computing Group\\
	}
	\url{http://hpc.uni-due.de/teaching}\\
	\vfill
	\textbf{Prof. Dr. Jens Krüger}\\
	\hpcPeople
\end{flushright}
\end{minipage}
\\
\vspace{-3.5mm}
\hrule
}

\hpcHeader

\let\origthesubsection\thesubsection
\setcounter{section}{0}

\vspace{0.5cm}

\begin{center}
	\LARGE
	\textbf{\hpcClass}\\
\end{center}

\vspace{0.2cm}

\ifhpcGerman % GERMAN STUFF

\normalfont
\begin{center}
	\Large
	{\textbf{\hpcSolution{Musterlösung für}{} Übungsblatt \hpcSheetNumber \ (\total{totalPoints} Punkte)}}
\end{center}

\vspace{0.2cm}

\hpcSolution{
\hrule height 0.07ex
}{
Alle Abgaben müssen in \Moodle bis \textbf{\hpcDeadline{}} hochgeladen werden, anderenfalls wird das Übungsblatt mit 0 Punkten bewertet. Abgaben, die nicht kompilierbar sind oder falsch benannte Dateien enthalten, bekommen ebenfalls 0 Punkte.
Beispielhafte (Konsolen-)Ausgaben sind häufig Teil der Aufgabe. Stimmen Ihre Lösungen nicht mit den beispielhaften Ausgaben überein, erhalten Sie nur sehr wenige Punkte. Bitte fügen Sie keine weiteren, nicht in der Aufgabe geforderten Ausgaben hinzu.
Packen Sie alle Dateien Ihrer Lösung in \textbf{eine}
\href{http://de.wikipedia.org/wiki/ZIP-Dateiformat}{Zip-Datei}.
Innerhalb dieser Zip-Datei dürfen keine Unterverzeichnisse angelegt werden. Der Dateiname muss wie folgt aussehen: \textbf{\texttt{u\hpcSheetNumber\_Vorname\_Nachname.zip}}.
}

\newcommand{\ex}[2]{
\addtocounter{totalPoints}{#2}
\section{#1 (#2 Punkte)}
}

\else % ENGLISH STUFF

\normalfont
\begin{center}
	\Large
	{\textbf{\hpcSolution{Sample Solution for}{} Assignment \hpcSheetNumber \ (\total{totalPoints} Points)}}
\end{center}

\vspace{0.2cm}

\hpcSolution{
\hrule height 0.07ex
}{
All assignments are to be uploaded to \Moodle on \textbf{\hpcDeadline{}}.
Assignments that are not uploaded to Moodle before the deadline will
not be graded (0 points). Assignments that do not compile also
receive 0 points.
Often example inputs and outputs are provided for you
to test your programs; assignments that do not process these examples
correctly will receive very few points.
Please do not add any additional output to your program than what is
requested.
Submit your solution as a \textbf{single} \href{http://en.wikipedia.org/wiki/Zip_(file_format)}{zip-archive} named as follows: \textbf{\texttt{a\hpcSheetNumber\_FirstName\_LastName.zip}}.

Include the entire project folder in the archive and for assignments also consisting of theoretical tasks include in addition a text file or a scan/photo with the corresponding answers.
Please see assignment sheet 1 for details.
}

\newcommand{\ex}[2]{
\addtocounter{totalPoints}{#2}
\section{#1 (#2 Points)}
}

\fi

\vspace{1em}
%\hline
\vspace{1em}


\section{Hello Triangles}
The provided code in this exercise already integrates some functionality from the previous assignment like GLFW and GLEW initialization in classes inside the \texttt{Utils} folder. In addition the folder contains classes for matrices and vectors which you will need for this and following assignments. It is not needed to edit any of the source code inside that folder instead it is sufficient to just edit classes inside the \texttt{Hello\_Triangles} folder. Folder \texttt{res} in \texttt{Hello\_Triangles} contains files for your shader sources and you can use the \texttt{GLHelper} files for compiling the shader sources.

\ex{Colored triangles}{10}
\label{ex:triangles}

The provided example program \texttt{main.cpp} in folder \texttt{Hello\_Triangles} renders a simple white triangle as shown on the left in Figure~\ref{fig:triangles}. Extend this program so that its output matches \textbf{exactly} the rendering on the right in the same figure that shows three \textbf{equilateral} triangles with interpolated colors of pure red, magenta, yellow, blue, cyan, and green.

Study the matrix and vector classes inside \texttt{Utils} to perform transformations which will be essential for the upcoming exercises.

\begin{figure}
  \centering
  \includegraphics[width=0.4875\linewidth]{SimpleTriangle.png}
  \includegraphics[width=0.4875\linewidth]{Triforce.png}
  \caption{Rendering of a simple white triangle (left) and the \href{http://en.wikipedia.org/wiki/Triforce}{Triforce} rendered with smooth interpolated colors (right).}
  \label{fig:triangles}
\end{figure}




\ex{Transformations \& animations}{10}
\label{ex:animations}

Extend and modify your solution to Exercise~\ref{ex:triangles} so that each of the three equilateral triangles shown on the right in Figure~\ref{fig:triangles} is rotating at 45 degrees per second around one of the major axes. The upper red-magenta-yellow triangle should rotate around the $x$-axis aligned at the half of the triangle's height. The lower right yellow-cyan-green triangle should rotate around the $y$-axis aligned at half of the triangle's width and the lower left magenta-blue-cyan triangle should rotate around the $z$-axis where the triangle's centroid should be the center of its rotation. You should still use an orthographic projection as in the previous exercise. For better comparison you can watch the clip of the animated triangles on Moodle.

Finally, process user input in your program so that pressing the space bar toggles the animation to stop or to continue right at the position where it stopped. Additionally, the \enquote{\texttt{R}}-key should reset the animation to its first frame as displayed in Figure~\ref{fig:triangles} (right). \textbf{When the application starts the animation should be paused in the reset state.}


\end{document}
