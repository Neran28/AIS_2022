\documentclass[12pt, a4paper, oneside, ngerman]{article}

\newif\ifhpcGerman
\hpcGermanfalse

\usepackage[utf8]{inputenc}
\usepackage{geometry}
\geometry{a4paper, left=25mm, right=20mm, top=2.35cm, bottom=2.35cm}

\usepackage[T1]{fontenc}
\usepackage{totcount}
\usepackage{calc}
\usepackage{graphicx}
\usepackage{ifthen}
\usepackage[usenames]{color}
\usepackage{caption}
\usepackage{multicol}
\usepackage{lipsum}
\usepackage{listings}
\usepackage{courier}
%\usepackage[nolist]{acronym}
\usepackage[acronym]{glossaries} % do not use hyperref! that's why we include it before
\usepackage[hyphens]{url}
\usepackage[pdfborder={0 0 0}, colorlinks=true,urlcolor=blue,linkcolor=blue,citecolor=blue]{hyperref}
\usepackage{microtype}
\usepackage{tabu}
\usepackage{xspace}
\usepackage{fix-cm}
%\usepackage{pdflscape}
\usepackage{lscape}
\usepackage{enumitem}
%\usepackage{enumerate}
\usepackage{amsmath}
\usepackage{amssymb}
\usepackage{wasysym}
\usepackage[detect-weight=true, detect-family=true]{siunitx}
\usepackage{xfrac}
\usepackage{mathtools}
\newcommand{\abs}[1]{ \left\lvert#1\right\rvert} % absolute value: single vertical bars
\newcommand{\norm}[1]{\left\lVert#1\right\rVert} % norm: double vertical bars

\ifhpcGerman % GERMAN STUFF
\usepackage[ngerman]{babel}
\usepackage[babel, german=quotes]{csquotes}
%\usepackage[babel, german=guillemets]{csquotes}
\else
%\usepackage[english=british]{csquotes}
\usepackage[english=american]{csquotes}
\fi

\newif\ifhpcBonus % enables bonus points
\hpcBonusfalse
\newif\ifhpcFake % toggles preparatory (fake) and final exam (real)
\hpcFakefalse
\newif\ifhpcKnown % toggles known student name, matriculation no. etc.
\hpcKnownfalse
\newif\ifhpcSolution % toggles solution
\hpcSolutionfalse

\newcommand{\hpcGerman}[2]{\ifhpcGerman#1\else#2\fi}
\newcommand{\hpcBonus}[2]{\ifhpcBonus#1\else#2\fi}
\newcommand{\hpcFake}[2]{\ifhpcFake#1\else#2\fi}
\newcommand{\hpcKnown}[2]{\ifhpcKnown#1\else#2\fi}

% The solution commands take always two arguments. The solution text that should be displayed
% when solution mode is enabled and the other text that should be displayed in the normal mode.
% Use simple "\s{<solution text>}{<normal text>} for bold colored solutions.
\newcommand{\s}[2]{\ifhpcSolution\textcolor[rgb]{0.8,0.0,0.8}{\textbf{#1}}\else#2\fi}
%\newcommand{\s}[2]{\ifprintanswers\textcolor[rgb]{0.8,0.0,0.8}{\textbf{#1}}\else#2\fi} % exam solution stuff
% Use "\solution{<solution text>}{<normal text>} for subtle solutions.
\newcommand{\hpcSolution}[2]{\ifhpcSolution#1\else#2\fi}
%\newcommand{\hpcSolution}[2]{\ifprintanswers#1\else#2\fi} % exam solution stuff

\definecolor{dkgreen}{rgb}{0,0.4,0}
\definecolor{dkred}{rgb}{0.5,0,0}
\definecolor{lightgray}{rgb}{0.9,0.9,0.9}
\definecolor{mygray}{rgb}{0.5,0.5,0.5}
\definecolor{mymauve}{rgb}{0.58,0,0.82}

\lstdefinelanguage{out}{
%sensitive=false,
%alsoletter={.},
%moredelim=[s][\color{red}]{<}{>},
%moredelim=[s][\color{blue}]{[}{]},
%moredelim=[is][\color{orange}]{:}{:},
%keywords=[10]{...},
%keywordstyle=[10]{\color{magenta}},
}

\lstnewenvironment{stdout}
{\lstset{numbers=none, language=out, frame=none, backgroundcolor={\color{lightgray}}, basicstyle={\footnotesize\ttfamily}}}
{}

\lstset{
  % the following allows us to handle UTF-8 encoding issues, listings does not support UTF-8 code
  literate=
  {á}{{\'a}}1 {é}{{\'e}}1 {í}{{\'i}}1 {ó}{{\'o}}1 {ú}{{\'u}}1
  {Á}{{\'A}}1 {É}{{\'E}}1 {Í}{{\'I}}1 {Ó}{{\'O}}1 {Ú}{{\'U}}1
  {à}{{\`a}}1 {è}{{\'e}}1 {ì}{{\`i}}1 {ò}{{\`o}}1 {ò}{{\`u}}1
  {À}{{\`A}}1 {È}{{\'E}}1 {Ì}{{\`I}}1 {Ò}{{\`O}}1 {Ò}{{\`U}}1
  {ä}{{\"a}}1 {ë}{{\"e}}1 {ï}{{\"i}}1 {ö}{{\"o}}1 {ü}{{\"u}}1
  {Ä}{{\"A}}1 {Ë}{{\"E}}1 {Ï}{{\"I}}1 {Ö}{{\"O}}1 {Ü}{{\"U}}1
  {â}{{\^a}}1 {ê}{{\^e}}1 {î}{{\^i}}1 {ô}{{\^o}}1 {û}{{\^u}}1
  {Â}{{\^A}}1 {Ê}{{\^E}}1 {Î}{{\^I}}1 {Ô}{{\^O}}1 {Û}{{\^U}}1
  {œ}{{\oe}}1 {Œ}{{\OE}}1 {æ}{{\ae}}1 {Æ}{{\AE}}1 {ß}{{\ss}}1
  {ç}{{\c c}}1 {Ç}{{\c C}}1 {ø}{{\o}}1 {å}{{\r a}}1 {Å}{{\r A}}1
  {€}{{\EUR}}1 {£}{{\pounds}}1,
  %extendedchars=true,
  %backgroundcolor=\color{white},    % choose the background color; you must add \usepackage{color} or \usepackage{xcolor}
  numbers=left,                      % where to put the line-numbers; possible values are (none, left, right)
  numbersep=5pt,                     % how far the line-numbers are from the code
  numberstyle=\scriptsize\ttfamily\color{mygray},   % the style that is used for the line-numbers
  language=Java,                     % the language of the code
  basicstyle=\scriptsize\ttfamily, % the size of the fonts that are used for the code
  showspaces=false,                  % show spaces adding particular underscores
  showstringspaces=false,            % underline spaces within strings
  showtabs=false,                    % show tabs within strings adding particular underscores
  showlines=true,                    % show empty lines at the end of the listing
  %stepnumber=2,                     % the step between two line-numbers. If it's 1, each line will be numbered
  frame=leftline,                    % adds a frame around the code
  rulecolor=\color{black},           % if not set, the frame-color may be changed on line-breaks within not-black text (e.g. commens (green here))
  tabsize=2,                         % sets default tabsize to 2 spaces
  stringstyle=\color{mymauve},       % string literal style
  captionpos=b,                      % sets the caption-position to bottom
  breaklines=true,                   % sets automatic line breaking
  breakatwhitespace=false,           % sets if automatic breaks should only happen at whitespace
  keywordstyle=\color{blue},         % keyword style
  commentstyle=\color{dkgreen},      % comment style
  stringstyle=\color{dkred},         % string literal style
  escapeinside={\%*}{*)},            % if you want to add LaTeX within your code
  %title=\lstname,                   % show the filename of files included with \lstinputlisting; also try caption instead of title
  morekeywords={in, vec3, vec4, *,...} % if you want to add more keywords to the set
}
\lstloadlanguages{Java}
\DeclareCaptionFont{green}{\color{green}}
\DeclareCaptionFont{white}{\color{white}}
\DeclareCaptionFormat{listing}{\colorbox[cmyk]{0.43, 0.35, 0.35,0.01}{\parbox{\textwidth}{\hspace{15pt}#1#2#3}}}

\captionsetup[lstlisting]{format=listing, labelfont=white,
                         textfont=white, singlelinecheck=false, margin=0pt,
                         font={footnotesize}}



\newcommand{\hpcSheetNumber}{5}
\newcommand{\hpcDeadline}{June 23, 2020 at 11:55pm}

%%% define class and exam details in header file!

%\hpcSolutiontrue

\begin{document}

\rmfamily

\newtotcounter{totalPoints}

\newcommand{\Moodle}{\href{https://moodle2.uni-due.de/course/view.php?id=10062}{Moodle}\xspace}
\newcommand{\hpcClass}{Advanced Image Synthesis\xspace}
\newcommand{\hpcTerm}{Summer Term 2021\xspace}
\newcommand{\hpcPeople}{\textbf{Andrea Iannella}\xspace}

%\newcommand{\class}[1]{\texttt{#1}}

% cover page header
\newcommand{\hpcHeader}{
\noindent
\begin{minipage}[b][3.5cm]{0.39\textwidth}
	\includegraphics[width=6cm]{../uni-due}\\
	\vfill
%	\hpcGerman{
%	\s{Lösung}{} \hpcFake{Probeklausur}{Klausur} \hpcTerm\
%	}{
%	\s{Solution}{} \hpcFake{Preparatory Exam}{Final Exam} \hpcTerm\
%	}
	\hpcTerm
	%\\ \hpcExamDate
\end{minipage}
\hfill
\begin{minipage}[b][3.5cm]{0.59\textwidth}
\begin{flushright}
	\hpcGerman{
	Informatik und Angewandte Kognitionswissenschaft\\
	Lehrstuhl für Hochleistungsrechnen\\
	}{
	Computer Science and Applied Cognitive Science\\
	High Performance Computing Group\\
	}
	\url{http://hpc.uni-due.de/teaching}\\
	\vfill
	\textbf{Prof. Dr. Jens Krüger}\\
	\hpcPeople
\end{flushright}
\end{minipage}
\\
\vspace{-3.5mm}
\hrule
}

\hpcHeader

\let\origthesubsection\thesubsection
\setcounter{section}{0}

\vspace{0.5cm}

\begin{center}
	\LARGE
	\textbf{\hpcClass}\\
\end{center}

\vspace{0.2cm}

\ifhpcGerman % GERMAN STUFF

\normalfont
\begin{center}
	\Large
	{\textbf{\hpcSolution{Musterlösung für}{} Übungsblatt \hpcSheetNumber \ (\total{totalPoints} Punkte)}}
\end{center}

\vspace{0.2cm}

\hpcSolution{
\hrule height 0.07ex
}{
Alle Abgaben müssen in \Moodle bis \textbf{\hpcDeadline{}} hochgeladen werden, anderenfalls wird das Übungsblatt mit 0 Punkten bewertet. Abgaben, die nicht kompilierbar sind oder falsch benannte Dateien enthalten, bekommen ebenfalls 0 Punkte.
Beispielhafte (Konsolen-)Ausgaben sind häufig Teil der Aufgabe. Stimmen Ihre Lösungen nicht mit den beispielhaften Ausgaben überein, erhalten Sie nur sehr wenige Punkte. Bitte fügen Sie keine weiteren, nicht in der Aufgabe geforderten Ausgaben hinzu.
Packen Sie alle Dateien Ihrer Lösung in \textbf{eine}
\href{http://de.wikipedia.org/wiki/ZIP-Dateiformat}{Zip-Datei}.
Innerhalb dieser Zip-Datei dürfen keine Unterverzeichnisse angelegt werden. Der Dateiname muss wie folgt aussehen: \textbf{\texttt{u\hpcSheetNumber\_Vorname\_Nachname.zip}}.
}

\newcommand{\ex}[2]{
\addtocounter{totalPoints}{#2}
\section{#1 (#2 Punkte)}
}

\else % ENGLISH STUFF

\normalfont
\begin{center}
	\Large
	{\textbf{\hpcSolution{Sample Solution for}{} Assignment \hpcSheetNumber \ (\total{totalPoints} Points)}}
\end{center}

\vspace{0.2cm}

\hpcSolution{
\hrule height 0.07ex
}{
All assignments are to be uploaded to \Moodle on \textbf{\hpcDeadline{}}.
Assignments that are not uploaded to Moodle before the deadline will
not be graded (0 points). Assignments that do not compile also
receive 0 points.
Often example inputs and outputs are provided for you
to test your programs; assignments that do not process these examples
correctly will receive very few points.
Please do not add any additional output to your program than what is
requested.
Submit your solution as a \textbf{single} \href{http://en.wikipedia.org/wiki/Zip_(file_format)}{zip-archive} named as follows: \textbf{\texttt{a\hpcSheetNumber\_FirstName\_LastName.zip}}.

Include the entire project folder in the archive and for assignments also consisting of theoretical tasks include in addition a text file or a scan/photo with the corresponding answers.
Please see assignment sheet 1 for details.
}

\newcommand{\ex}[2]{
\addtocounter{totalPoints}{#2}
\section{#1 (#2 Points)}
}

\fi

\vspace{1em}
%\hline
\vspace{1em}


\ex{Diffuse details}{15}

Reuse your submission for assignment 4 as your codebase for this assignment.
Your implementation for this assignment should be exclusively based on \textit{Phong} shading.
Submit your solution for the exercises on this sheet according to the submission guidelines.

In order to increase detail in our teapot scene add texture mapping capabilities to your program. Use the textures displayed in Figure~\ref{fig:diffuseTextures} for the diffuse (\texttt{Stones\_Diffuse.png}) and specular (\texttt{Stones\_Specular.png}) illumination components of the ground plane. The textures should span the entire ground plane. Make sure to enable bilinear texture interpolation for mini- and magnification. For loading the texture files into your program you can use the provided \texttt{stb\_image.cpp} and \texttt{stb\_image.h} files. Also extend your \texttt{GLProgram} class for handling \texttt{sampler2D} uniforms. \\

\textbf{As in the previous assignment: Do not change the initial viewing angle and light source position of the scene for your submission!}\\

You may want to compare your results with Figure~\ref{fig:diffuseScene}.

\begin{figure}
  \centering
  \includegraphics[width=0.4875\linewidth]{Stones_Diffuse.png}
  \includegraphics[width=0.4875\linewidth]{Stones_Specular.png}
  \caption{Diffuse (left) and specular (right) textures for the ground plane. All textures can be downloaded on \Moodle.}
  \label{fig:diffuseTextures}
\end{figure}

\begin{figure}
  \centering
  \includegraphics[width=0.7\linewidth]{TeapotDiffuseDetails.png}
  \caption{Diffuse and specular textures applied to the teapot scene.}
  \label{fig:diffuseScene}
\end{figure}



\ex{Normal details}{15}

Increase the small-scale detail level of your scene even further by implementing bump mapping. Use the normal maps shown in Figure~\ref{fig:normalTextures} to enhance the ground plane (\texttt{Stones\_Normals.png}) and the teapot (\texttt{HPC\_Normals.png}). Do not forget to enable the texture repeat mode for the teapot's texture. If implemented correctly the stones should appear to stick out of the ground plane's surface under steep viewing angles and appropriate lighting conditions and the HPC logo should engrave the blue teapot.

For better comparison you can watch the animation sequences on \Moodle and compare your results with Figure~\ref{fig:normalScene}.

\begin{figure}
  \centering
  \includegraphics[width=0.325\linewidth]{Stones_Normals.png}
  \includegraphics[width=0.65\linewidth]{HPC_Normals.png}
  \caption{Normal maps for the ground plane (left) and teapot (right). All textures can be downloaded on \Moodle.}
  \label{fig:normalTextures}
\end{figure}

\begin{figure}
  \centering
  \includegraphics[width=0.7\linewidth]{TeapotNormalDetails.png}
  \caption{Diffuse, specular, and normal map textures applied to the teapot scene.}
  \label{fig:normalScene}
\end{figure}


\end{document}
